% Options for packages loaded elsewhere
\PassOptionsToPackage{unicode}{hyperref}
\PassOptionsToPackage{hyphens}{url}
\PassOptionsToPackage{dvipsnames,svgnames,x11names}{xcolor}
%
\documentclass[
  letterpaper,
  DIV=11,
  numbers=noendperiod]{scrartcl}

\usepackage{amsmath,amssymb}
\usepackage{iftex}
\ifPDFTeX
  \usepackage[T1]{fontenc}
  \usepackage[utf8]{inputenc}
  \usepackage{textcomp} % provide euro and other symbols
\else % if luatex or xetex
  \usepackage{unicode-math}
  \defaultfontfeatures{Scale=MatchLowercase}
  \defaultfontfeatures[\rmfamily]{Ligatures=TeX,Scale=1}
\fi
\usepackage{lmodern}
\ifPDFTeX\else  
    % xetex/luatex font selection
\fi
% Use upquote if available, for straight quotes in verbatim environments
\IfFileExists{upquote.sty}{\usepackage{upquote}}{}
\IfFileExists{microtype.sty}{% use microtype if available
  \usepackage[]{microtype}
  \UseMicrotypeSet[protrusion]{basicmath} % disable protrusion for tt fonts
}{}
\makeatletter
\@ifundefined{KOMAClassName}{% if non-KOMA class
  \IfFileExists{parskip.sty}{%
    \usepackage{parskip}
  }{% else
    \setlength{\parindent}{0pt}
    \setlength{\parskip}{6pt plus 2pt minus 1pt}}
}{% if KOMA class
  \KOMAoptions{parskip=half}}
\makeatother
\usepackage{xcolor}
\setlength{\emergencystretch}{3em} % prevent overfull lines
\setcounter{secnumdepth}{-\maxdimen} % remove section numbering
% Make \paragraph and \subparagraph free-standing
\ifx\paragraph\undefined\else
  \let\oldparagraph\paragraph
  \renewcommand{\paragraph}[1]{\oldparagraph{#1}\mbox{}}
\fi
\ifx\subparagraph\undefined\else
  \let\oldsubparagraph\subparagraph
  \renewcommand{\subparagraph}[1]{\oldsubparagraph{#1}\mbox{}}
\fi


\providecommand{\tightlist}{%
  \setlength{\itemsep}{0pt}\setlength{\parskip}{0pt}}\usepackage{longtable,booktabs,array}
\usepackage{calc} % for calculating minipage widths
% Correct order of tables after \paragraph or \subparagraph
\usepackage{etoolbox}
\makeatletter
\patchcmd\longtable{\par}{\if@noskipsec\mbox{}\fi\par}{}{}
\makeatother
% Allow footnotes in longtable head/foot
\IfFileExists{footnotehyper.sty}{\usepackage{footnotehyper}}{\usepackage{footnote}}
\makesavenoteenv{longtable}
\usepackage{graphicx}
\makeatletter
\def\maxwidth{\ifdim\Gin@nat@width>\linewidth\linewidth\else\Gin@nat@width\fi}
\def\maxheight{\ifdim\Gin@nat@height>\textheight\textheight\else\Gin@nat@height\fi}
\makeatother
% Scale images if necessary, so that they will not overflow the page
% margins by default, and it is still possible to overwrite the defaults
% using explicit options in \includegraphics[width, height, ...]{}
\setkeys{Gin}{width=\maxwidth,height=\maxheight,keepaspectratio}
% Set default figure placement to htbp
\makeatletter
\def\fps@figure{htbp}
\makeatother

\KOMAoption{captions}{tableheading}
\usepackage{authblk}
\makeatletter
\@ifpackageloaded{caption}{}{\usepackage{caption}}
\AtBeginDocument{%
\ifdefined\contentsname
  \renewcommand*\contentsname{Table of contents}
\else
  \newcommand\contentsname{Table of contents}
\fi
\ifdefined\listfigurename
  \renewcommand*\listfigurename{List of Figures}
\else
  \newcommand\listfigurename{List of Figures}
\fi
\ifdefined\listtablename
  \renewcommand*\listtablename{List of Tables}
\else
  \newcommand\listtablename{List of Tables}
\fi
\ifdefined\figurename
  \renewcommand*\figurename{Figure}
\else
  \newcommand\figurename{Figure}
\fi
\ifdefined\tablename
  \renewcommand*\tablename{Table}
\else
  \newcommand\tablename{Table}
\fi
}
\@ifpackageloaded{float}{}{\usepackage{float}}
\floatstyle{ruled}
\@ifundefined{c@chapter}{\newfloat{codelisting}{h}{lop}}{\newfloat{codelisting}{h}{lop}[chapter]}
\floatname{codelisting}{Listing}
\newcommand*\listoflistings{\listof{codelisting}{List of Listings}}
\makeatother
\makeatletter
\makeatother
\makeatletter
\@ifpackageloaded{caption}{}{\usepackage{caption}}
\@ifpackageloaded{subcaption}{}{\usepackage{subcaption}}
\makeatother
\ifLuaTeX
  \usepackage{selnolig}  % disable illegal ligatures
\fi
\usepackage{bookmark}

\IfFileExists{xurl.sty}{\usepackage{xurl}}{} % add URL line breaks if available
\urlstyle{same} % disable monospaced font for URLs
\hypersetup{
  pdftitle={Introduction aux mégadonnées en sciences sociales (FAS 1001)},
  pdfauthor={Hiver 2025 ~},
  colorlinks=true,
  linkcolor={blue},
  filecolor={Maroon},
  citecolor={Blue},
  urlcolor={red},
  pdfcreator={LaTeX via pandoc}}

\title{Introduction aux mégadonnées en sciences sociales (FAS 1001)}
\author{Hiver 2025 ~}
\date{}

\begin{document}
\maketitle

\textbf{Horaire du cours:} Jeudi, 8h30 - 11h29

\textbf{Local de cours:} pavillon Marie-Victorin local G-415\_555B

\textbf{Disponibilité:} Sur rendez-vous

\textbf{Courriel:}
\href{mailto:lomf0@ulaval.ca}{\nolinkurl{lomf0@ulaval.ca}}

\rule{\textwidth}{2pt}

\section{Prérequis}\label{pruxe9requis}

Ce cours ne demande aucun prérequis en statistiques avancées ou en
programmation, mais avoir des connaissances au préalable demeure un
atout. Il est fortement conseillé de suivre en parallèle le cours
\href{https://florencevdubois.github.io/teaching.html}{FAS 1003} -
\emph{Visualisation des données}, car la visualisation graphique
demeurera un élément important dans la présentation de vos travaux. Ce
cours priorise un enseignement avec des ateliers afin de permettre aux
étudiant-es de se pratiquer à la programmation, de poser des questions
au chargé de cours et d'effectuer leur travail de mi-session et de fin
de session.

\section{Description du cours et
objectifs}\label{description-du-cours-et-objectifs}

Jamais autant de données n'ont été disponibles pour comprendre les
comportements humains. De nouveaux outils de recherche nous permettent
désormais de quantifier des données difficilement analysables
auparavant, tels que de larges corpus de textes, des images ou des
audios en provenance de vidéos. Néanmoins, comment collecter, traiter et
analyser ces nouvelles données? Comment combiner ces nouvelles données
avec celles déjà largement utilisées en sciences sociales comme les
sondages? Quels sont les enjeux techniques et éthiques que soulèvent
l'utilisation de ces données en recherche dans le contexte d'un
développement important de l'intelligence artificielle et de
l'application croissante de l'apprentissage par machines en sciences
sociales? Ce cours aborde ces nombreux enjeux avec une
\textbf{\emph{approche pratique}} à l'utilisation des mégadonnées en
sciences sociales.

À la fin de ce cours, les objectifs suivants seront remplis:

\begin{itemize}
\item
  Avoir une connaissance globale des différentes sources de données
  disponibles pour étudier les phénomènes humains.
\item
  Développer l'autonomie nécessaire pour collecter, gérer et analyser
  quantitativement des bases de données et les présenter avec un projet
  de recherche les mobilisant.
\item
  Faire preuve d'une compréhension des enjeux entourant la mobilisation
  et l'utilisation de volumineuses bases de données en sciences
  sociales.
\end{itemize}

\section{Pédagogie}\label{puxe9dagogie}

Le langage de programmation utilisé pour ce cours est \texttt{R}. Ce
dernier est téléchargeable gratuitement
\href{https://cran.r-project.org/}{ici}. Bien que plusieurs options
soient possibles, il vous est également demandé de télécharger
\texttt{RStudio} \href{https://posit.co/download/rstudio-desktop/}{ici}.
RStudio est l'environnement de développement intégré de prédilection
pour coder en \texttt{R}. Il vous permettra d'éditer et de déboguer plus
facilement votre code en plus de vous donner les outils nécessaires pour
transformer, prévisualiser et analyser vos données efficacement.
Certains étudiant-es ont signalé être plus à l'aise d'utiliser le
langage de programmation \texttt{Python}. Si vous vous sentez plus à
l'aise d'utiliser ce dernier, il est possible de le faire lors de vos
travaux de session. Cependant, les capacités du chargé de cours à vous
aider seront plus limitées. Cela est à prendre en considération. Les
étudiant-es devront, si ce n'est pas déjà fait, créer un compte
\href{https://github.com/}{\texttt{GitHub}} lors du premier cours, afin
de remettre leurs codes de travaux pratiques et de travaux de session.
\texttt{GitHub} est un service web d'hébergement et de gestion de
développement de logiciels qui est largement utilisé en industrie pour
le partage de codes. Cela permettra de socialiser les étudiant-es à son
utilisation.

\begin{itemize}
\item
  Le bon déroulement de ce cours nécessite que vous ayez un ordinateur,
  notamment pour les nombreux ateliers pratiques lors des classes. Si
  vous n'avez pas d'ordinateur, une solution sera trouvée, afin que vous
  puissiez participer aux activités en classe.
\item
  Généralement, le cours comporte une heure et demie d'enseignement
  magistral, une pause de 15 minutes et une heure d'atelier en classe.
  Le cours magistral abordera les lectures et autres ressources
  obligatoires à consulter avant le cours ainsi que le contenu du jour.
  Il est attendu une présence des étudiant-es aux cours, car nos
  ateliers en classe seront importants pour comprendre la matière puis
  pour avancer dans ses travaux de session.
\item
  Il y aura également la présence de plusieurs personnes invitées ayant
  des expériences concrètes de recherche avec des mégadonnées à
  l'intérieur et à l'extérieur du milieu académique. Les présentations
  seront discutées au début du cours en plus des lectures/vidéos et
  autres sources que vous devez consulter avant chacun d'entre eux.
  Lorsqu'un-e invité-e sera présent-e, la partie magistrale du cours
  sera légèrement moins longue. Les présentations des invité-es dureront
  environ 30 minutes avec une période de questions de 15 minutes.
\end{itemize}

\section{Évaluations}\label{uxe9valuations}

Sur un total de 100 points:

\subsection{Participation en classe
(10/100)}\label{participation-en-classe-10100}

\textbf{10\% des points} seront alloués à \textbf{la participation en
classe}. L'entraide entre les étudiant-es est encouragée (évaluation et
correction de code sur \texttt{GitHub} ou lors des ateliers).

\subsection{Travaux pratiques (20/100)}\label{travaux-pratiques-20100}

\textbf{20\% des points} seront alloués \textbf{aux travaux pratiques}.
Les étudiant-es devront remettre 4 travaux pratiques qui permettront
d'évaluer leur évolution à différentes étapes de la session. Les travaux
pratiques devront être mis sur votre compte \texttt{GitHub} \textbf{une
journée avant le cours suivant, soit lundi 23h59}. Vous pouvez commencer
ou même terminer vos travaux pratiques lors des ateliers qui sont
également des périodes pour avancer vos travaux de session. Les jours de
remise seront dans le calendrier du cours.

\begin{itemize}
\item
  Introduction à \texttt{GitHub} (\textbf{5 points})
\item
  Combiner des données de sondages (\textbf{5 points})
\item
  Analyse de données textuelles (\textbf{5 points})
\item
  Webscraping (\textbf{5 points})
\end{itemize}

\subsection{Travaux de session (70/100)}\label{travaux-de-session-70100}

\subsubsection{Travail de mi-session
(15/100)}\label{travail-de-mi-session-15100}

\textbf{15\% des points} seront alloués à la remise d'un \textbf{plan de
recherche de votre travail de session de 5 pages}. Le plan de recherche
permettra d'avoir un regard du chargé de cours sur l'évolution du
travail de session et de corriger la trajectoire de ce dernier si
nécessaire. Le plan de recherche consistera principalement en la
présentation de votre question de recherche, les raisons motivant votre
recherche, les données que vous comptez utiliser et la ou les méthode-s
mobilisée-s. Ce travail peut également être la base de votre travail de
session final.

\subsubsection{Travail de session
(40/100)}\label{travail-de-session-40100}

\textbf{40\% des points} seront alloués à \textbf{votre travail de
session individuel pouvant aller de 10 à 15 pages}. Ce dernier
consistera en une recherche complète avec présentation de la question de
recherche, les raisons motivant cette dernière, la présentation des
données utilisées, la ou les méthode-s mobilisée-s, l'analyse des
données et de la présentation des conclusions de l'étude. La ou les
méthode-s utilisée-s pour analyser les données n'ont pas besoin d'être
poussée-s, une plus grande attention sera portée sur les données
collectées, la transformation effectuée sur ces dernières et leur
présentation.

\subsection{Présentation orale
(15/1000)}\label{pruxe9sentation-orale-151000}

\textbf{15\% des points} seront alloués à la présentation orale de
\textbf{votre travail de session}. La présentation de votre travail de
session vous amènera à parler un peu plus de vos données, des
conclusions de vos recherches, mais également du cheminement par lequel
vous êtes passé, afin d'arriver aux dernières étapes du projet. Les
présentations seront de 10 à 15 minutes avec 10 minutes de questions de
l'audience et s'échelonneront sur deux périodes du cours. Un support
visuel est requis pour les présentations orales. Il est attendu que tous
les étudiant-es soient présent-es aux présentations orales.

\section{Ouvrage obligatoire}\label{ouvrage-obligatoire}

Le livre de référence du cours sera celui du Professeur Rohan Alexander
de l'Université de Toronto
\href{https://tellingstorieswithdata.com/}{\emph{Telling Stories with
Data}} et disponible gratuitement sur son site web.

\section{Ouvrage ressource}\label{ouvrage-ressource}

Pour un livre ressource en méthodes quantitatives, il est conseillé de
consulter le livre du Professeur Vincent Arel Bundock
\href{https://pum.umontreal.ca/catalogue/analyse_causale_et_methodes_quantitatives}{\emph{Analyse
causale et méthodes quantitatives}} disponible gratuitement sur le site
des Presses de l'Université de Montréal.

\section{Calendrier}\label{calendrier}

\subsection{Cours 1 (9 janvier 2024) : Introduction et présentation du
plan de
cours.}\label{cours-1-9-janvier-2024-introduction-et-pruxe9sentation-du-plan-de-cours.}

\begin{itemize}
\item
  Lectures pour le prochain cours: Lire les chapitres
  \href{https://tellingstorieswithdata.com/03-workflow.html}{3} et
  \href{https://tellingstorieswithdata.com/04-writing_research.html}{4}.
\item
  Bonus: \textbf{Vous pouvez lire si intéressé(e)} les chapitres
  \href{https://tellingstorieswithdata.com/01-introduction.html}{1} et
  \href{https://tellingstorieswithdata.com/02-drinking_from_a_fire_hose.html}{2}.
\end{itemize}

\subsection{Cours 2 (16 janvier 2024) : Outils de communication et de
collaboration en
recherche.}\label{cours-2-16-janvier-2024-outils-de-communication-et-de-collaboration-en-recherche.}

\begin{itemize}
\item
  Description: Introduction à \texttt{GitHub} et \texttt{Quarto}.
\item
  Lectures pour le prochain cours: Lire la section 8.3 du chapitre
  \href{https://tellingstorieswithdata.com/08-hunt.html}{8}. Lire
  également Breton, Cutler, Lachance et Mierke-Zatwarnicki (2017),
  \emph{Telephone versus online survey modes for election studies:
  Comparing Canadian public opinion and vote choice in the 2015 federal
  election}, dans la \emph{Revue Canadienne de Science Politique}.
  Visionner ce
  \href{https://www.youtube.com/watch?app=desktop&v=5SybR3KiBMw&ab_channel=TLDRBusiness}{vidéo}
  qui montre un exemple cocasse d'une recherche avec sondage qui
  souligne les forces et les limites de ce type de données.
\item
  Bonus: \textbf{Vous pouvez lire si intéressé(e)} Zaller et Feldman
  (1992), \emph{A Simple Theory of the Survey Response: Answering
  Questions versus Revealing Preferences}, dans \emph{l'American Journal
  of Political Science} sur les réponses données lors des sondages. Vous
  pouvez également lire Ansolabehere, Rodden et Snyder (2008), \emph{The
  Strength of Issues: Using Multiple Measures to Gauge Preference
  Stability, Ideological Constraint, and Issue Voting}, sur les échelles
  de mesure pour réduire les erreurs en analyse de sondages. Ces
  lectures seront utiles pour comprendre la matière du prochain cours.
\end{itemize}

\subsection{Cours 3 (23 janvier 2024) : Données de
sondages.}\label{cours-3-23-janvier-2024-donnuxe9es-de-sondages.}

\begin{itemize}
\item
  Description: Utilisation, avantages et inconvénients, structuration de
  bases de données, transformation de variables, variables latentes,
  expériences par sondage et quasi-expériences.
\item
  Lecture pour le prochain cours: Lire Grimmer, Roberts, et Stewart
  (2021). \emph{Machine Learning for Social Science: An Agnostic
  Approach.}, dans \emph{Annual Review of Political Science}.
\end{itemize}

\subsection{Cours 4 (30 janvier 2024) : Machine
Learning.}\label{cours-4-30-janvier-2024-machine-learning.}

\begin{itemize}
\item
  Description: Entraînement de modèles, algorithmes et place de l'IA en
  sciences sociales.
\item
  Invitée: Professeure
  \href{https://www.catherineouellet.com/}{Catherine Ouellet} du
  département de science politique de l'Université de Montréal et
  co-créatrice de l'application Datagotchi.
\item
  Lectures pour le prochain cours: Lire le texte de Grimmer, et Stewart
  (2013) \emph{Text as Data: The Promise and Pitfalls of Automatic
  Content Analysis Methods for Political Texts} dans \emph{Political
  Analysis}. Lire également le chapitre
  \href{https://tellingstorieswithdata.com/16-text.html}{17}.
\end{itemize}

\subsection{Cours 5 (6 février 2024) : Analyses textuelles automatisées
partie
1.}\label{cours-5-6-fuxe9vrier-2024-analyses-textuelles-automatisuxe9es-partie-1.}

\begin{itemize}
\tightlist
\item
  Description: Données tirées de texte, analyses du dictionnaires,
  analyses supervisées.
\end{itemize}

\subsection{Cours 6 (13 février 2024) : Analyses textuelles automatisées
partie
2.}\label{cours-6-13-fuxe9vrier-2024-analyses-textuelles-automatisuxe9es-partie-2.}

\begin{itemize}
\item
  Description: Analyses non-supervisées, word embedding, machine
  learning et autres.
\item
  Lectures pour le prochain cours: Lire le texte de Chetty et collègues
  (2022), \emph{Social capital II: determinants of economic
  connectedness} dans \emph{Nature}. Bien important de lire
  \textbf{\emph{Social capital II}} et non I.
\item
  Bonus: \textbf{Vous pouvez lire si intéressé(e)} Barbera (2015),
  \emph{Birds of the Same Feather Tweet Together: Bayesian Ideal Point
  Estimation Using Twitter Data}, dans \emph{Political Analysis}; Chetty
  et collègues (2022), \emph{Social capital I: measurement and
  associations with economic mobility} dans \emph{Nature}; Guinaudeau,
  Munger et Votta (2022), \emph{Fifteen seconds of fame: TikTok and the
  supply side of social video}, dans \emph{Computational Communication
  Research}; Nyhan et collègues (2023), \emph{Like-minded sources on
  Facebook are prevalent but not polarizing} dans \emph{Nature}.
\end{itemize}

\newpage

\subsection{Cours 7 (20 février 2024) : Analyses de réseaux
sociaux.}\label{cours-7-20-fuxe9vrier-2024-analyses-de-ruxe9seaux-sociaux.}

\begin{itemize}
\item
  Description: Analyses possibles, enjeux et futur pour ces types de
  données en sciences sociales.
\item
  Invitée: Anne Imouza, doctorante au département de science politique
  de l'Université McGill et qui a utilisé des données de
  \texttt{Twitter} maintenant \texttt{X} pour son mémoire de maîtrise.
\end{itemize}

\subsection{Cours 8 (27 février 2024) : Atelier pour travailler sur le
projet de
mi-session.}\label{cours-8-27-fuxe9vrier-2024-atelier-pour-travailler-sur-le-projet-de-mi-session.}

\subsection{(5 mars 2024): Semaine de
lectures**}\label{mars-2024-semaine-de-lectures}

\begin{itemize}
\tightlist
\item
  Lecture pour le prochain cours: Lire le chapitre
  \href{https://tellingstorieswithdata.com/07-gather.html}{7}. Lire
  également l'article de Luscombe, Dick and Walby (2020),
  \emph{Algorithmic thinking in the public interest: navigating
  technical, legal, and ethical hurdles to web scraping in the social
  sciences}, dans \emph{Quality \& Quantity}.
\end{itemize}

\subsection{Cours 9 (12 mars 2024) : Scraping de
données.}\label{cours-9-12-mars-2024-scraping-de-donnuxe9es.}

\begin{itemize}
\item
  Description: Webscraping (données tirées du web), API.
\item
  Invité: \href{https://benguinaudeau.com/}{Benjamin Guinaudeau},
